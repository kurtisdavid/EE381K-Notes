\documentclass[psamsfonts, 12pt]{amsart}
%
%-------Packages---------
%
\usepackage[h margin=1 in, v margin=1 in]{geometry}
\usepackage{amssymb,amsfonts}
\usepackage[all,arc]{xy}
\usepackage{tikz-cd}
\usepackage{enumerate}
\usepackage{mathrsfs}
\usepackage{amsthm}
\usepackage{mathpazo}
\usepackage{float}
\usepackage[backend=biber]{biblatex}
\addbibresource{bibliography.bib}
%\usepackage{charter} %another font
%\usepackage{eulervm} %Vakil font
\usepackage{yfonts}
\usepackage{mathtools}
\usepackage{enumitem}
\usepackage{mathrsfs}
\usepackage{fourier-orns}
\usepackage[all]{xy}
\usepackage{hyperref}
\usepackage{url}
\usepackage{mathtools}
\usepackage{graphicx}
\usepackage{pdfsync}
\usepackage{mathdots}
\usepackage{calligra}
\usepackage{import}
\usepackage{xifthen}
\usepackage{pdfpages}
\usepackage{transparent}

\newcommand{\incfig}[2]{%
    \fontsize{48pt}{50pt}\selectfont
    \def\svgwidth{\columnwidth}
    \scalebox{#2}{\input{#1.pdf_tex}}
}
%
\usepackage{tgpagella}
\usepackage[T1]{fontenc}
%
\usepackage{listings}
\usepackage{color}

\definecolor{dkgreen}{rgb}{0,0.6,0}
\definecolor{gray}{rgb}{0.5,0.5,0.5}
\definecolor{mauve}{rgb}{0.58,0,0.82}

\lstset{frame=tb,
  language=Matlab,
  aboveskip=3mm,
  belowskip=3mm,
  showstringspaces=false,
  columns=flexible,
  basicstyle={\small\ttfamily},
  numbers=none,
  numberstyle=\tiny\color{gray},
  keywordstyle=\color{blue},
  commentstyle=\color{dkgreen},
  stringstyle=\color{mauve},
  breaklines=true,
  breakatwhitespace=true,
  tabsize=3
  }
%
%--------Theorem Environments--------
%
\newtheorem{thm}{Theorem}[section]
\newtheorem*{thm*}{Theorem}
\newtheorem{cor}[thm]{Corollary}
\newtheorem{prop}[thm]{Proposition}
\newtheorem{lem}[thm]{Lemma}
\newtheorem*{lem*}{Lemma}
\newtheorem{conj}[thm]{Conjecture}
\newtheorem{quest}[thm]{Question}
%
\theoremstyle{definition}
\newtheorem{defn}[thm]{Definition}
\newtheorem*{defn*}{Definition}
\newtheorem{defns}[thm]{Definitions}
\newtheorem{con}[thm]{Construction}
\newtheorem{exmp}[thm]{Example}
\newtheorem{exmps}[thm]{Examples}
\newtheorem{notn}[thm]{Notation}
\newtheorem{notns}[thm]{Notations}
\newtheorem{addm}[thm]{Addendum}
\newtheorem{exer}[thm]{Exercise}
%
\theoremstyle{remark}
\newtheorem{rem}[thm]{Remark}
\newtheorem*{claim}{Claim}
\newtheorem*{aside*}{Aside}
\newtheorem*{rem*}{Remark}
\newtheorem*{hint*}{Hint}
\newtheorem*{note}{Note}
\newtheorem{rems}[thm]{Remarks}
\newtheorem{warn}[thm]{Warning}
\newtheorem{sch}[thm]{Scholium}
%
%--------Macros--------
\renewcommand{\qedsymbol}{$\blacksquare$}
\renewcommand{\sl}{\mathfrak{sl}}
\newcommand{\Bord}{\mathsf{Bord}}
\renewcommand{\hom}{\mathsf{Hom}}
\renewcommand{\emptyset}{\varnothing}
\renewcommand{\O}{\mathscr{O}}
\newcommand{\R}{\mathbb{R}}
\newcommand{\ib}[1]{\textbf{\textit{#1}}}
\newcommand{\Q}{\mathbb{Q}}
\newcommand{\Z}{\mathbb{Z}}
\newcommand{\N}{\mathbb{N}}
\newcommand{\C}{\mathbb{C}}
\newcommand{\A}{\mathbb{A}}
\newcommand{\F}{\mathbb{F}}
\newcommand{\M}{\mathcal{M}}
\newcommand{\dbar}{\overline{\partial}}
\newcommand{\zbar}{\overline{z}}
\renewcommand{\S}{\mathbb{S}}
\newcommand{\V}{\vec{v}}
\newcommand{\RP}{\mathbb{RP}}
\newcommand{\CP}{\mathbb{CP}}
\newcommand{\B}{\mathcal{B}}
\newcommand{\GL}{\mathsf{GL}}
\newcommand{\SL}{\mathsf{SL}}
\newcommand{\SP}{\mathsf{SP}}
\newcommand{\SO}{\mathsf{SO}}
\newcommand{\SU}{\mathsf{SU}}
\newcommand{\gl}{\mathfrak{gl}}
\newcommand{\g}{\mathfrak{g}}
\newcommand{\Bun}{\mathsf{Bun}}
\newcommand{\inv}{^{-1}}
\newcommand{\bra}[2]{ \left[ #1, #2 \right] }
\newcommand{\set}[1]{\left\lbrace #1 \right\rbrace}
\newcommand{\abs}[1]{\left\lvert#1\right\rvert}
\newcommand{\norm}[1]{\left\lVert#1\right\rVert}
\newcommand{\transv}{\mathrel{\text{\tpitchfork}}}
\newcommand{\defeq}{\vcentcolon=}
\newcommand{\enumbreak}{\ \\ \vspace{-\baselineskip}}
\let\oldexists\exists
\renewcommand\exists{\oldexists~}
\let\oldL\L
\renewcommand\L{\mathfrak{L}}
\makeatletter
\newcommand{\tpitchfork}{%
  \vbox{
    \baselineskip\z@skip
    \lineskip-.52ex
    \lineskiplimit\maxdimen
    \m@th
    \ialign{##\crcr\hidewidth\smash{$-$}\hidewidth\crcr$\pitchfork$\crcr}
  }%
}
\makeatother
\newcommand{\bd}{\partial}
\newcommand{\lang}{\begin{picture}(5,7)
\put(1.1,2.5){\rotatebox{45}{\line(1,0){6.0}}}
\put(1.1,2.5){\rotatebox{315}{\line(1,0){6.0}}}
\end{picture}}
\newcommand{\rang}{\begin{picture}(5,7)
\put(.1,2.5){\rotatebox{135}{\line(1,0){6.0}}}
\put(.1,2.5){\rotatebox{225}{\line(1,0){6.0}}}
\end{picture}}
\DeclareMathOperator{\id}{id}
\DeclareMathOperator{\im}{Im}
\DeclareMathOperator{\codim}{codim}
\DeclareMathOperator{\coker}{coker}
\DeclareMathOperator{\supp}{supp}
\DeclareMathOperator{\inter}{Int}
\DeclareMathOperator{\sign}{sign}
\DeclareMathOperator{\sgn}{sgn}
\DeclareMathOperator{\indx}{ind}
\DeclareMathOperator{\alt}{Alt}
\DeclareMathOperator{\Aut}{Aut}
\DeclareMathOperator{\trace}{trace}
\DeclareMathOperator{\ad}{ad}
\DeclareMathOperator{\End}{End}
\DeclareMathOperator{\Ad}{Ad}
\DeclareMathOperator{\Lie}{Lie}
\DeclareMathOperator{\spn}{span}
\DeclareMathOperator{\dv}{div}
\DeclareMathOperator{\grad}{grad}
\DeclareMathOperator{\Sym}{Sym}
\DeclareMathOperator{\sheafhom}{\mathscr{H}\text{\kern -3pt {\calligra\large om}}\,}
\newcommand*\myhrulefill{%
   \leavevmode\leaders\hrule depth-2pt height 2.4pt\hfill\kern0pt}
\newcommand\niceending[1]{%
  \begin{center}%
    \LARGE \myhrulefill \hspace{0.2cm} #1 \hspace{0.2cm} \myhrulefill%
  \end{center}}
\newcommand*\sectionend{\niceending{\decofourleft\decofourright}}
\newcommand*\subsectionend{\niceending{\decosix}}
\def\upint{\mathchoice%
    {\mkern13mu\overline{\vphantom{\intop}\mkern7mu}\mkern-20mu}%
    {\mkern7mu\overline{\vphantom{\intop}\mkern7mu}\mkern-14mu}%
    {\mkern7mu\overline{\vphantom{\intop}\mkern7mu}\mkern-14mu}%
    {\mkern7mu\overline{\vphantom{\intop}\mkern7mu}\mkern-14mu}%
  \int}
\def\lowint{\mkern3mu\underline{\vphantom{\intop}\mkern7mu}\mkern-10mu\int}
%
%--------Hypersetup--------
%
\hypersetup{
    colorlinks,
    citecolor=black,
    filecolor=black,
    linkcolor=blue,
    urlcolor=blacksquare
}
%
%--------Solution--------
%
\newenvironment{solution}
  {\begin{proof}[Solution]}
  {\end{proof}}
%
%--------Graphics--------
%
%\graphicspath{ {images/} }

\begin{document}
%
\author{Kurtis David}
%
\title{Lecture 1: EE 381K - Convex Optimization}
%
\maketitle
%

\section{Class Policy Extras}

\underline{Linear Programming Books}:

\begin{enumerate}
    \item Bertsimas, Tsitsiklis "Introduction to Linear Optimization"
    \item R.J. Vanderbli "Linear Programming Foundation and Extension"
\end{enumerate}

\underline{Convex Optimization Books}:

\begin{enumerate}
    \item Ben Tal \& Nemirovski "Lectures on Modern Convex Opt
    \item Bertsekas, Nedic, Oz
\end{enumerate}

\section{Minimization}

General Formulation:

\begin{align*}
    &\text{minimize }f_0(x) \\
    &\text{subject to } f_i(x) \leq b_i(x)\ i = 1,...,m \\
    &f_0: \R^n \rightarrow \R \Rightarrow \text{Objective function} \\
    &f_i: \R^n \rightarrow \R\ i=1,...,m \Rightarrow  \text{Constraint function} \\
    &x \in \R^n \Rightarrow \text{ optimization/decision variable} \\
    &\hat{x}  \text{ feasible if } f_i(\hat{x}) \leq b_i(\hat{x})\\
    &x^*: \text{ optimal solution}\\
    &f_0(x^*) \leq f_0(\hat{x}) \forall \text{ feasible } \hat{x}
\end{align*}

\section{Linear Programming}

\begin{defn}

\underline{Linear programs} are one type of convex optimization problems, and they consist of:
\begin{itemize}
    \item Objective function $f_0$
    \item All the constraints $f_1,...,f_m$
\end{itemize}

If these are all linear functions, then it is a linear program.

\end{defn}

\[
    \min_{x_1,...,x_n} \sum_{j=1}^nc_jx_j\ |\ c_j \text{ cost of } x_j
\]
\[
    s.t. \sum_{j=1}^na_{ij}x_j \leq b_i \text{ for } i = 1,...,m \Rightarrow m \text{ Inequality constraints}
\]
\[
    \sum_{j=1}^nd_{ij}x_j = e_i \text{ for } i = 1,...,P \Rightarrow P \text{ Equality constraints}
\]

\underline{Example: Resource Allocation}

\textbf{Parameters} i.e. given variables

\begin{itemize}
    \item $n$: Number of activities $j=1,...,n$
    \item $m$: Number of Resources $i=1,...,m$
    \item $P_j$: Profit of activity $j$
    \item $b_i$: Amount of available resource $i$
    \item $a_{ij}$: amount of resource $i$ used by activity $j$
\end{itemize}

\textbf{Variables}
\begin{itemize}
    \item $x_j$ amount of activity $j$ selected for resource $i$
\end{itemize}

\textbf{Goal}

\begin{center}
 $\max \sum_{j=1}^nP_jx_j$
 
 s.t. $\sum_{j=1}^na_{ij}x_j \leq b_i$ for $i=1,...m$
 
 
 s.t. $x_j \geq 0$ for $j=1,...,m$
\end{center}

\underline{Example: Matching Problem}

We have $N$ people and $N$ tasks.

People indexed by $i=1,...N$

Tasks indexed by $j=1,...,N$

$a_{ij}$ cost of assigning task $j$ to person $i$

\[ x_{ij} = \begin{cases} 
      1 & \text{Assign task } j \text{ to person } i \\
      0 & \text{otherwise}
   \end{cases}
\]

\textbf{Goal}
\begin{center}
    minimize $\sum_{i=1}^N\sum_{j=1}^Na_{ij}x_{ij}$
    
    s.t. $\sum_{i=1}^Nx_{ij}=1$ $j=1,...,N$ 
    
    s.t. $\sum_{j=1}^Nx_{ij}=1$ $i=1,...,N$ 
    
    AND $x_{ij} \in \{0,1\}$
\end{center}

\textbf{Problem:} Feasible set is only $\{0,1\}$. 

Thus we should \textbf{relax} constraints such that $0 \leq x_{ij} \leq 1$. We will come back to this later.


\section{Vectorization}


Now let's vectorize these formulations.

\[
    c = \begin{bmatrix}
        c_1 \\
        \vdots \\
        c_n
    \end{bmatrix} \in \R^n\quad x = \begin{bmatrix}
        x_1 \\
        \vdots \\
        x_n
    \end{bmatrix} \in \R^n
\]
\[
    a_i = \begin{bmatrix}
        a_{i1} \\
        \vdots \\
        a_{in}
    \end{bmatrix} i = 1,..,m \quad d_i = \begin{bmatrix}
        d_{i1} \\
        \vdots \\
        d_{in}
    \end{bmatrix} i = 1,..,P
\]

\begin{center}
    minimize $c^Tx$
    
    s.t. $a_i^Tx \leq b_i\ i=1,...,m$
    
         $d_i^Tn = e_i\ i=1,...,P$
\end{center}

Now let 

\[
    A = \begin{bmatrix}
        a_{11} & a_{12} & ... & a_{1n}\\
        \vdots & \vdots & ... & \vdots\\
        a_{m1} & a_{m2} & ... & a_{mn}\\
    \end{bmatrix} \in \R^{mn} \quad D = \begin{bmatrix}
        d_{11} & d_{12} & ... & d_{1n}\\
        \vdots & \vdots & ... & \vdots\\
        d_{P1} & d_{P2} & ... & d_{Pn}\\
    \end{bmatrix} \in \R^{Pn}
\]

Now we can rewrite the constraints to be:

\begin{center}
    $Ax \leq b$
    
    $Dx = e$ 
\end{center}

\begin{rem}
Whenever using $\leq$ with vectors, this implies all elements in one vector are $\leq$ than their respective element (same idx) in the other vector.
\end{rem}

\section{Shit ton of Definitions}

\begin{defn}
$x$ is \underline{feasible} if it satisfies $Ax\leq b$ \& $Dx = e$
\end{defn}

\begin{defn}
\underline{feasible set} $S = \{ x\in\R^n |Ax\leq b$ \& $Dx = e \}$
\end{defn}

\begin{defn}
$x^*$ is \underline{optimal} if $c^Tx^* \leq c^Tx$ for any $x\in S$
\end{defn}

\begin{defn}
\underline{optimal value} $p^* = c^Tx^*$
\end{defn}

\begin{defn}
\underline{Unbounded LP} $p^* = c^Tx^*$
\end{defn}

\begin{defn}
\underline{Hyperplane}: Solution set of one linear equation with nonzero coeff. vector $a$ ($a_i\neq 0$) s.t. $a^Tx = b$
\end{defn}

\begin{defn}
\underline{Half Space}: Solution set of one linear inequality with nonzero coefficients. i.e. $a^Tx \leq b$.
\end{defn}

\begin{defn}
\underline{Subspace} Intersection of a set of hyperplanes. Or a solution to a system of equality equations.
\end{defn}

\begin{defn}
\underline{Polyhedron} Intersection of a set of half-spaces. Or a solution to a finite number of linear inequalities.
\end{defn}

\begin{defn}
\underline{Function set} Set of points where some $f$ has value $\alpha$. i.e. $f(x) = \alpha$. e.g. hyperplanes
\end{defn}

\section{Polyhedrons}

\[
    P = \{ x | Ax\leq B, Cx = d\}
\]

\begin{defn}
    \underline{Lineality space}: The lineality space of $P$ is defined as 
    
    \[
        L = \text{nullspace}( \begin{bmatrix}
        A \in \R^{mn}\\
        C \in \R^{Pn}
        \end{bmatrix}
    \]
\end{defn}

\begin{claim}
Let $x\in P$, $v\in L$. Then $x + v \in L$.
\end{claim}

\begin{proof}
Trivial. $x$ in the solution space, and $Av = 0 = Cv$
\end{proof}

\begin{defn}
\underline{Pointed polyhderon} A polyhderon $P$ with Lineality space $ L = \{0\}$. i.e. the null space of both $A$ and $C$ is trivial.
\end{defn}

$\Rightarrow$ A polyhderon is pointed if it doesn't contain a \textit{line}.

\underline{Example 1}

a half space $\{ x  a^T x \leq b \}$.

Only when $x\in\R$ is this a pointed polyhedron.

\underline{Example 2}

a half space $\{ x  -1 \leq a^T x \leq 1\}$.

Only when $x\in\R$ is this a pointed polyhedron.

\underline{Example 3}

a half space $\{ x  |x|\leq 1 |y| \leq 1\}$.

$S = \{(0,0,z) | z\in\R\}$

therefore always not pointed.

\underline{Example 4}

$\{ x | 1^Tx = 1, x\geq 0\}$.

YEET it is pointed





\end{document}